\documentclass[conference]{IEEEtran}

\usepackage{blindtext}
\usepackage{graphicx}
\usepackage{hyperref}
%\usepackage{caption}
%\usepackage{subcaption}

\usepackage{cite}
% Loading the cite package will
% result in citation numbers being automatically sorted and properly
% "compressed/ranged". e.g., [1], [9], [2], [7], [5], [6] without using
% cite.sty will become [1], [2], [5]--[7], [9] using cite.sty. cite.sty's
% \cite will automatically add leading space, if needed. 

% correct bad hyphenation here
\hyphenation{op-tical net-works semi-conduc-tor}


\begin{document}

% can use linebreaks \\ within to get better formatting as desired
\title{StudyPal: A Website for Study Groups}

\author{\IEEEauthorblockN{James Beavers}
\IEEEauthorblockA{Department of Computer Science\\
North Carolina State University\\
Raleigh, North Carolina, USA\\
jjbeaver@ncsu.edu}
\and
\IEEEauthorblockN{Anthony Elliott}
\IEEEauthorblockA{Department of Computer Science\\
North Carolina State University\\
Raleigh, North Carolina, USA\\
anthony\_elliott@ncsu.edu}}


\maketitle


\begin{abstract}
\blindtext[1]
\end{abstract}

\IEEEpeerreviewmaketitle



\section{Introduction}
StudyPal is a website we designed and created to help university students form study groups for their classes.

%todo: might be better to merge these subsections later, I created them just to make sure I got everything

\subsection{Task Domain}
Students often form study groups for a specific class in which they help each other with exam preparation, readings, or assignments.
These study groups are formalized meetings set up ahead of time, in contrast to unplanned meetings.
Meeting length can vary from one hour to several hours.
Group sizes are typically small, with 2-5 participants.
Groups often consist of systematically going through a study guide for the class, either provided by the class instructor or created by the group participants.


\subsection{Target Users}
StudyPal is intended for use by both undergraduate and graduate students at North Carolina State University (NCSU).
If successful at NCSU then we plan to extend StudyPal to students at other universities.
StudyPal currently only uses the English language and thus only expected to be used by English-speaking students.


\subsection{Representative Scenarios}

\subsubsection{Joining a Group}
Users enter the website and enter their user information to create an account.
After logging in to the website, the user inputs their class schedule and selects which of their current classes they are interested in joining a study group for.
StudyPal then displays all current study groups for that class that are still looking for more members.
The user scrolls through the list of groups and selects one to view the details of.
The user looks at the desired qualifications of group members and hits 'Apply' if interested in joining the group.
The group creator then sees that someone has requested group access and can choose to accept or reject this applicant.

If there are no groups for this class or the user does not want to join any of the existing groups then they can create a group themselves.

\subsubsection{Creating a Group}
A user who wants to create a study group first logs in to the StudyPal website, and hits the 'Create a Group' button displayed on multiple pages.
They enter basic information about the group, such as course and frequency of meetings.
Fields such as time and place can be skipped if not yet decided on.
The user then enters qualifications such as 'have read all chapters', or 'attends all classes' that every group member should have.
This creates the group and the user waits for other students to see this group and apply to join.
The user can then accept or reject students who apply to join the group.

\subsubsection{Group Notifications}
%todo: put more detail here
%todo: will we implement a better notification system (e.g. button for 'Email all participants')?
Each participant in a StudyPal group can see all other email addresses of the other participants and can use email to send updates to the group.



\section{Related Work}
A description of related work, including related systems and papers from the academic literature. Related work encompasses interactive systems in the task domain, interactive systems for related tasks, and general HCI research. Reference any external material you rely on including papers, books, commercial systems, Web pages, and so forth. Highlight the influences of this material on your work. 

\emph{The report demonstrates depth of knowledge of the area, in both theoretical and practical terms.}

\section{System Design}
A description of the design of the system. You should also describe its evolution through the design process, along with the influence of user testing and formative evaluation techniques on the design decisions you made. Artifacts (e.g., notes from interviews, sketches, early mock-ups, etc.) maybe included in the written report as an appendix. 

\emph{The system clearly reflect users' concerns, as demonstrated by the extensive use of relevant HCI techniques for discovery and design. User feedback has been taken into account in the design, as illustrated by specific examples (e.g., quotes from users and descriptions of design changes). Best practices in HCI development have been followed.}

% user interviews
% sketches
% development
% user feedback (where I'm the user)

We wanted to create a very simple user interface that could be quickly used by an expert.
We also wanted our website to be easy to use for novices, that is, we wanted novices to be able to complete basic tasks in the website without requiring outside help.
The specific tasks that we wanted to support are joining an existing group, creating a new group, and sending notifications to the group members.

We first started by interviewing several students who have previously used study groups for their classes in order to better determine the scope of our system.

After extracting requirements from the interviews, we sketched mockups of the website using pencil and paper.
%todo These sketches are included in ____
The sketches helped us identify challenges that we would face.

One challenge was how to integrate the user's schedule into our system.
One user said that ``even though courses appear on my schedule, I don't always have to go to them and would still be available to meet.''
This highlights the issue that a schedule may not be accurate for all users and thus we don't want to hide groups that meet during these times in case the user actually wants to attend those meetings.
However, many users would \emph{not} want to see groups that conflict with classes.
To resolve this issue we decided to indicate conflicting times by dimming the background color of those groups when trying to find a group.
%todo implement the dimming


In interviews, one participant said that ``the level of distraction increases with the number of people in the group.''
The same participant said that 5 study group members was the upper limit for productivity, however, other interview participants said they frequently had study groups with around 5 members.
This shows that a generally small group is more desirable, but further research is needed to see how group size impacts study group results.
We decided to prominently display the current group size for potential applicants so that they could make their own decision if they wanted to join a group with 5 or more members.

Another issue was group member requirements.
We discovered in interviews that study groups are usually formed via personal connections and many subjective decisions were made to determine who to invite to the group.
One participant said that all study group members ``need a common background'', for example, each member should have at least passing familiarity with each section of material.
Defining this 'common background' is an example of such subjective decisions.
We did not want to implement complex qualification algorithms without basing decisions on solid research, so we allow group leaders to create subjective qualifications and decide themselves.
Group leaders enter desired qualifications of group members, other students can apply after reading through the qualifications, and then the leader reviews each application and can either accept or reject them.


Frequency of study groups is another issue that emerged from the interviews.
Some students typically formed a study group that only met once per exam while others met on a regular basis.
We extracted discrete intervals of once, weekly, and monthly for study groups.





\section{Analytical Interface Evaluation}
An analytical evaluation of your interface, using modeling techniques you judge to be most appropriate. 
 
\emph{A specific analytical technique modeling technique is correctly applied; its choice is well-justified and the results are explained in detail. The analytical technique is such that it provides predictions or explanations of actual user performance. Insights from the analytical evaluation are discussed. The relevance of the modeling results to usability are made clear.}

% use the KLM
 
In this section we provide an analytical evaluation of StudyPal.
Specifically, we describe why we chose the KLM model [cite], how we applied the model to three use cases of StudyPal, and what these results mean in terms of the usability of StudyPal.

\subsection{Why KLM?}
[cite]

% explain why using the KLM
 % sufficient for our needs
 % a more detailed analysis would benefit from the NGOMSL because it factors in learning time
 % KLM limitations: 
   % errors
   % learning
   % functionality
   % recall
   % concentration
   % fatigue
   % acceptability
 % we don't care much about the missing pieces above (except learning)

\subsection{Application of KLM}
We describe how we applied the KLM to the three main use cases of StudyPal.
All three use cases, or tasks, assume that the user is already logged in to StudyPal and located at the home page.

\subsubsection{Joining a Group}

\subsubsection{Creating a Group}

\subsubsection{Send Notification to Group}



\subsection{Usability Implications}


\section{Empirical Interface Evaluation}
An empirical, formative evaluation of your interface. This should be run with at least four users, classmates or non-classmates. The evaluation records both quantitative and qualitative information. 

\emph{The formative evaluation is well-designed to give information about the usability of the system. It also provides information that allows for comparison with the analytical evaluation; that comparison is made. Results are discussed in enough detail that their generality is clear, suggesting how well the system would work if it could be deployed in the real world and a full-scale empirical evaluation carried out. In other words, it's clear how well the system works.}

% needed in second column of first page if using \IEEEpubid
%\IEEEpubidadjcol

% An example of a floating figure using the graphicx package.
% Note that \label must occur AFTER (or within) \caption.
% For figures, \caption should occur after the \includegraphics.
% Note that IEEEtran v1.7 and later has special internal code that
% is designed to preserve the operation of \label within \caption
% even when the captionsoff option is in effect. However, because
% of issues like this, it may be the safest practice to put all your
% \label just after \caption rather than within \caption{}.
%
% Reminder: the "draftcls" or "draftclsnofoot", not "draft", class
% option should be used if it is desired that the figures are to be
% displayed while in draft mode.
%
%\begin{figure}[!t]
%\centering
%\includegraphics[width=2.5in]{myfigure}
% where an .eps filename suffix will be assumed under latex, 
% and a .pdf suffix will be assumed for pdflatex; or what has been declared
% via \DeclareGraphicsExtensions.
%\caption{Simulation Results}
%\label{fig_sim}
%\end{figure}

% Note that IEEE typically puts floats only at the top, even when this
% results in a large percentage of a column being occupied by floats.


% An example of a double column floating figure using two subfigures.
% (The subfig.sty package must be loaded for this to work.)
% The subfigure \label commands are set within each subfloat command, the
% \label for the overall figure must come after \caption.
% \hfil must be used as a separator to get equal spacing.
% The subfigure.sty package works much the same way, except \subfigure is
% used instead of \subfloat.
%
%\begin{figure*}[!t]
%\centerline{\subfloat[Case I]\includegraphics[width=2.5in]{subfigcase1}%
%\label{fig_first_case}}
%\hfil
%\subfloat[Case II]{\includegraphics[width=2.5in]{subfigcase2}%
%\label{fig_second_case}}}
%\caption{Simulation results}
%\label{fig_sim}
%\end{figure*}
%
% Note that often IEEE papers with subfigures do not employ subfigure
% captions (using the optional argument to \subfloat), but instead will
% reference/describe all of them (a), (b), etc., within the main caption.


% An example of a floating table. Note that, for IEEE style tables, the 
% \caption command should come BEFORE the table. Table text will default to
% \footnotesize as IEEE normally uses this smaller font for tables.
% The \label must come after \caption as always.
%
%\begin{table}[!t]
%% increase table row spacing, adjust to taste
%\renewcommand{\arraystretch}{1.3}
% if using array.sty, it might be a good idea to tweak the value of
% \extrarowheight as needed to properly center the text within the cells
%\caption{An Example of a Table}
%\label{table_example}
%\centering
%% Some packages, such as MDW tools, offer better commands for making tables
%% than the plain LaTeX2e tabular which is used here.
%\begin{tabular}{|c||c|}
%\hline
%One & Two\\
%\hline
%Three & Four\\
%\hline
%\end{tabular}
%\end{table}


% Note that IEEE does not put floats in the very first column - or typically
% anywhere on the first page for that matter. Also, in-text middle ("here")
% positioning is not used. Most IEEE journals use top floats exclusively.
% Note that, LaTeX2e, unlike IEEE journals, places footnotes above bottom
% floats. This can be corrected via the \fnbelowfloat command of the
% stfloats package.



\section{Conclusion}
\blindtext





% if have a single appendix:
%\appendix[Proof of the Zonklar Equations]
% or
%\appendix  % for no appendix heading
% do not use \section anymore after \appendix, only \section*
% is possibly needed

% use appendices with more than one appendix
% then use \section to start each appendix
% you must declare a \section before using any
% \subsection or using \label (\appendices by itself
% starts a section numbered zero.)
%


\appendices
\section{Proof of the First Zonklar Equation}
\blindtext


% Can use something like this to put references on a page
% by themselves when using endfloat and the captionsoff option.
\ifCLASSOPTIONcaptionsoff
  \newpage
\fi



% trigger a \newpage just before the given reference
% number - used to balance the columns on the last page
% adjust value as needed - may need to be readjusted if
% the document is modified later
%\IEEEtriggeratref{8}
% The "triggered" command can be changed if desired:
%\IEEEtriggercmd{\enlargethispage{-5in}}

% references section

% can use a bibliography generated by BibTeX as a .bbl file
% BibTeX documentation can be easily obtained at:
% http://www.ctan.org/tex-archive/biblio/bibtex/contrib/doc/
% The IEEEtran BibTeX style support page is at:
% http://www.michaelshell.org/tex/ieeetran/bibtex/
%\bibliographystyle{IEEEtran}
% argument is your BibTeX string definitions and bibliography database(s)
%\bibliography{IEEEabrv,../bib/paper}
%
% <OR> manually copy in the resultant .bbl file
% set second argument of \begin to the number of references
% (used to reserve space for the reference number labels box)
\begin{thebibliography}{1}

\bibitem{IEEEhowto:kopka}
H.~Kopka and P.~W. Daly, \emph{A Guide to \LaTeX}, 3rd~ed.\hskip 1em plus
  0.5em minus 0.4em\relax Harlow, England: Addison-Wesley, 1999.

\end{thebibliography}

% biography section
% 
% If you have an EPS/PDF photo (graphicx package needed) extra braces are
% needed around the contents of the optional argument to biography to prevent
% the LaTeX parser from getting confused when it sees the complicated
% \includegraphics command within an optional argument. (You could create
% your own custom macro containing the \includegraphics command to make things
% simpler here.)
%\begin{biography}[{\includegraphics[width=1in,height=1.25in,clip,keepaspectratio]{mshell}}]{Michael Shell}
% or if you just want to reserve a space for a photo:

\begin{IEEEbiography}[{\includegraphics[width=1in,height=1.25in,clip,keepaspectratio]{picture}}]{John Doe}
\blindtext
\end{IEEEbiography}

% You can push biographies down or up by placing
% a \vfill before or after them. The appropriate
% use of \vfill depends on what kind of text is
% on the last page and whether or not the columns
% are being equalized.

%\vfill

% Can be used to pull up biographies so that the bottom of the last one
% is flush with the other column.
%\enlargethispage{-5in}




% that's all folks
\end{document}


